\documentclass[a4paper, 12pt]{article}
\usepackage[top=2cm, bottom=2cm, left=2.5cm, right=2.5cm]{geometry}
\usepackage[utf8]{inputenc}

\begin{document}

% O texto vem aqui.O texto vem aqui.O texto vem aqui.O texto vem aqui.O texto vem aqui.
% O texto vem aqui.O texto vem aqui.O texto vem aqui.O texto vem aqui.O texto vem aqui.
% O texto vem aqui.O texto vem aqui.O texto vem aqui.O texto vem aqui.O texto vem aqui.

% O texto vem aqui.O texto vem aqui.O texto vem aqui.O texto vem aqui.O texto vem aqui.
% O texto vem aqui.O texto vem aqui.O texto vem aqui.O texto vem aqui.O texto vem aqui.
% O texto vem aqui.O texto vem aqui.O texto vem aqui.O texto vem aqui.O texto vem aqui.

% Equa\c c\~ao polinomial do 2$^\circ$ grau.

% Vov\'o Vov\^o

Equação polinomial do 2º grau.

Uma equação da forma $ax^2 + bx + c = 0$, $a \neq 0$ será chamada de equação polinomial do 2º grau.

% Com $$ o LaTex quebra o texto destacando a fórmula em uma linha separada

Uma equação da forma $$ax^2 + bx + c = 0$$, $a \neq 0$ será chamada de equação polinomial do 2º grau.

A solução dessa equação é dada por
$$ x = \frac{-b \pm \sqrt{b^2 - 4ac}}{2a} $$

\begin{center}\
\textbf{Equação polinomial do 2º grau.}
\end{center}

\begin{flushright}
\textit{Equação polinomial do 2º grau.}
\end{flushright}

\begin{flushleft}
\underline{Equação polinomial do 2º grau.}
\end{flushleft}

\begin{flushleft}
\underline{\textbf{\textit{Equação polinomial do 2º grau.}}}
\end{flushleft}

\end{document}