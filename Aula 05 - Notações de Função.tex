\documentclass[a4paper]{article}
\usepackage[top=2cm, bottom=2cm, left=2.5cm, right=2.5cm]{geometry}
\usepackage[utf8]{inputenc}
\usepackage{amsmath,amsfonts,amssymb}
\DeclareMathOperator {\sen}{sen}
% O comando acima serve para facilitar a escrita quando os nomes em inglês e português diferem

\begin{document}

\begin{enumerate}
	\item Seja a função $ f: \mathbb{R} \to \mathbb{R} $ definida por 
	$ f(x) = \dfrac{1}{2} x^2 - 2x +1 $
	\begin{enumerate}
		\item Esboce o gráfico da função
	\end{enumerate}
	
% A barra dupla "\\" no código abaixo serve para quebrar a linha	
	
	\item $$ f(x) = 
		\begin{cases}
			
			x^2 - 1; \, \textrm{se } x \geq 1 \\
			x - 3; \, \textrm{se } -1 \leq x < 1 \\
			2x + 1; \, \textrm{se } x < -1
		
		\end{cases}
	$$	
	\begin{enumerate}
		\item Esboce o gráfico da função
	\end{enumerate}
	
	\item $ f(x) = 2^{x-1} $
		\begin{enumerate}
		\item Esboce o gráfico da função
	\end{enumerate}
	
	\item Seja a função $ f: \mathbb{R}^*_+ \to \mathbb{R} $ definida por 
	$ f(x) = \log_2 x $
	\begin{enumerate}
		\item Esboce o gráfico da função
	\end{enumerate}
	
	\item Seja a função $ f: \mathbb{R}^*_+ \to \mathbb{R} $ definida por 
	$ f(x) = \ln x $
	\begin{enumerate}
		\item Esboce o gráfico da função
	\end{enumerate}
	
	\item Seja a função $ f(x) = \cos x $
	\begin{enumerate}
		\item Esboce o gráfico da função
	\end{enumerate}
	
	\item Seja a função $ f(x) = \textrm{sen} \, x $
	\begin{enumerate}
		\item Esboce o gráfico da função
	\end{enumerate}
	
	\item Seja a função $ f(x) = \sen \left(x - \frac{\pi}{2}\right) $
	\begin{enumerate}
		\item Esboce o gráfico da função
	\end{enumerate}
	
\end{enumerate}

\end{document}