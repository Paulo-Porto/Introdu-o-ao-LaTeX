\documentclass[a4paper]{article}
\usepackage[top=2cm, bottom=2cm, left=2.5cm, right=2.5cm]{geometry}
\usepackage[utf8]{inputenc}
\usepackage{amsmath,amsfonts,amssymb}

\begin{document}

\begin{enumerate}

	\item Seja o segmento $ \overline{AB} $. A partir dele podemos definir os seguimentos
	orientados $ \overrightarrow{AB} $ e $ \overrightarrow{BA} $
	
	\item Sejam os vetores $ \vec{u} = (1; -1; 2) $ e $ \vec{v} = (2; 5; -4) $, calcule:
	\begin{enumerate}
		\item $ \vec{u} \cdot \vec{v} $
		\item $ \langle \vec{u},\vec{v} \rangle $
		\item $ \vec{u} \times \vec{v} $
		\item verifique se $ \vec{u} \perp \vec{v} $
		% O item abaixo  serve apenas para ilustrar o uso das Letras Gregas
		\item sejam os planos $ \alpha : x - 2y + 6z - 3 = 0 $ e $ \beta : x - 2y + 6z - 3 = 0 $
		\item sejam os vetores $ \vec{u} = (x_0, y_0, z_0) $ e $ \vec{v} = (x_1,y_1,z_1) $. Temos que:
		$$		
		\vec{u} \times \vec{v} = 
		\begin{vmatrix}
			\vec{i} & \vec{j} & \vec{k} \\
			x_0 & y_0 & z_0 \\
			x_1 & y_1 & z_1 \\
		\end{vmatrix}		
		$$
	\end{enumerate}
				
\end{enumerate}

\end{document}