\documentclass[a4paper]{article}
\usepackage[top=2cm, bottom=2cm, left=2.5cm, right=2.5cm]{geometry}
\usepackage[utf8]{inputenc}
\usepackage{amsmath,amsfonts,amssymb}
\usepackage[portuguese]{babel}

\begin{document}

\begin{enumerate}

	\item A tabela abaixo representa as derivadas básicas.
	$ \\ $
	
	\begin{tabular}{cc} %As opcoes aqui sao "l", "r" ou "c" para alinhar dentro da tabela
	
			Função & Derivada \\
			$ f(x) = x^n $ & $ f'(x) = nx^{n-1} $ \\
			$ f(x) = \log_a x $ & $ \dfrac{1}{(\ln a)x} $ \\
	
	\end{tabular}
	
	$ \\ $

	\begin{tabular}{|c|c|} %As opcoes aqui sao "l", "r" ou "c" para alinhar dentro da tabela
	
			Função & Derivada \\
			$ f(x) = x^n $ & $ f'(x) = nx^{n-1} $ \\
			$ f(x) = \log_a x $ & $ \dfrac{1}{(\ln a)x} $ \\
	
	\end{tabular}

	$ \\ $
	
	\begin{tabular}{|c|c|} %As opcoes aqui sao "l", "r" ou "c" para alinhar dentro da tabela
	
			\hline
			Função & Derivada \\ \hline
			$ f(x) = x^n $ & $ f'(x) = nx^{n-1} $ \\ \hline
			$ f(x) = \log_a x $ & $ \dfrac{1}{(\ln a)x} $ \\ \hline
	
	\end{tabular}

	$ \\ $
	\begin{center} % Comando para centralizar a tabela
		\begin{tabular}{|c|c|} %As opcoes aqui sao "l", "r" ou "c" para alinhar dentro da tabela
	
				\hline
				Função & Derivada \\ \hline
				$ f(x) = x^n $ & $ f'(x) = nx^{n-1} $ \\ \hline
				$ f(x) = \log_a x $ & $ \dfrac{1}{(\ln a)x} $ \\ \hline
	
		\end{tabular}
	\end{center}
	
	$ \\ $
	\item A Tabela \ref{minha-tabela-01} abaixo representa as derivadas básicas.
	\begin{table}[!htb] % Comando para nomearmos a depois citar a tabela
	\centering
		\begin{tabular}{|c|c|} %As opcoes aqui sao "l", "r" ou "c" para alinhar dentro da tabela
	
				\hline
				Função & Derivada \\ \hline
				$ f(x) = x^n $ & $ f'(x) = nx^{n-1} $ \\ \hline
				$ f(x) = \log_a x $ & $ \dfrac{1}{(\ln a)x} $ \\ \hline
	
		\end{tabular}
		\caption{Tabela Básica de Derivadas.}
		\label{minha-tabela-01}
	\end{table}

\end{enumerate}

\end{document}